% Options for packages loaded elsewhere
\PassOptionsToPackage{unicode}{hyperref}
\PassOptionsToPackage{hyphens}{url}
%
\documentclass[
]{article}
\usepackage{amsmath,amssymb}
\usepackage{lmodern}
\usepackage{ifxetex,ifluatex}
\ifnum 0\ifxetex 1\fi\ifluatex 1\fi=0 % if pdftex
  \usepackage[T1]{fontenc}
  \usepackage[utf8]{inputenc}
  \usepackage{textcomp} % provide euro and other symbols
\else % if luatex or xetex
  \usepackage{unicode-math}
  \defaultfontfeatures{Scale=MatchLowercase}
  \defaultfontfeatures[\rmfamily]{Ligatures=TeX,Scale=1}
\fi
% Use upquote if available, for straight quotes in verbatim environments
\IfFileExists{upquote.sty}{\usepackage{upquote}}{}
\IfFileExists{microtype.sty}{% use microtype if available
  \usepackage[]{microtype}
  \UseMicrotypeSet[protrusion]{basicmath} % disable protrusion for tt fonts
}{}
\makeatletter
\@ifundefined{KOMAClassName}{% if non-KOMA class
  \IfFileExists{parskip.sty}{%
    \usepackage{parskip}
  }{% else
    \setlength{\parindent}{0pt}
    \setlength{\parskip}{6pt plus 2pt minus 1pt}}
}{% if KOMA class
  \KOMAoptions{parskip=half}}
\makeatother
\usepackage{xcolor}
\IfFileExists{xurl.sty}{\usepackage{xurl}}{} % add URL line breaks if available
\IfFileExists{bookmark.sty}{\usepackage{bookmark}}{\usepackage{hyperref}}
\hypersetup{
  pdftitle={Foundations of AI - Assignment 1},
  pdfauthor={Garrett Ramela},
  hidelinks,
  pdfcreator={LaTeX via pandoc}}
\urlstyle{same} % disable monospaced font for URLs
\usepackage[margin=1in]{geometry}
\usepackage{color}
\usepackage{fancyvrb}
\newcommand{\VerbBar}{|}
\newcommand{\VERB}{\Verb[commandchars=\\\{\}]}
\DefineVerbatimEnvironment{Highlighting}{Verbatim}{commandchars=\\\{\}}
% Add ',fontsize=\small' for more characters per line
\usepackage{framed}
\definecolor{shadecolor}{RGB}{248,248,248}
\newenvironment{Shaded}{\begin{snugshade}}{\end{snugshade}}
\newcommand{\AlertTok}[1]{\textcolor[rgb]{0.94,0.16,0.16}{#1}}
\newcommand{\AnnotationTok}[1]{\textcolor[rgb]{0.56,0.35,0.01}{\textbf{\textit{#1}}}}
\newcommand{\AttributeTok}[1]{\textcolor[rgb]{0.77,0.63,0.00}{#1}}
\newcommand{\BaseNTok}[1]{\textcolor[rgb]{0.00,0.00,0.81}{#1}}
\newcommand{\BuiltInTok}[1]{#1}
\newcommand{\CharTok}[1]{\textcolor[rgb]{0.31,0.60,0.02}{#1}}
\newcommand{\CommentTok}[1]{\textcolor[rgb]{0.56,0.35,0.01}{\textit{#1}}}
\newcommand{\CommentVarTok}[1]{\textcolor[rgb]{0.56,0.35,0.01}{\textbf{\textit{#1}}}}
\newcommand{\ConstantTok}[1]{\textcolor[rgb]{0.00,0.00,0.00}{#1}}
\newcommand{\ControlFlowTok}[1]{\textcolor[rgb]{0.13,0.29,0.53}{\textbf{#1}}}
\newcommand{\DataTypeTok}[1]{\textcolor[rgb]{0.13,0.29,0.53}{#1}}
\newcommand{\DecValTok}[1]{\textcolor[rgb]{0.00,0.00,0.81}{#1}}
\newcommand{\DocumentationTok}[1]{\textcolor[rgb]{0.56,0.35,0.01}{\textbf{\textit{#1}}}}
\newcommand{\ErrorTok}[1]{\textcolor[rgb]{0.64,0.00,0.00}{\textbf{#1}}}
\newcommand{\ExtensionTok}[1]{#1}
\newcommand{\FloatTok}[1]{\textcolor[rgb]{0.00,0.00,0.81}{#1}}
\newcommand{\FunctionTok}[1]{\textcolor[rgb]{0.00,0.00,0.00}{#1}}
\newcommand{\ImportTok}[1]{#1}
\newcommand{\InformationTok}[1]{\textcolor[rgb]{0.56,0.35,0.01}{\textbf{\textit{#1}}}}
\newcommand{\KeywordTok}[1]{\textcolor[rgb]{0.13,0.29,0.53}{\textbf{#1}}}
\newcommand{\NormalTok}[1]{#1}
\newcommand{\OperatorTok}[1]{\textcolor[rgb]{0.81,0.36,0.00}{\textbf{#1}}}
\newcommand{\OtherTok}[1]{\textcolor[rgb]{0.56,0.35,0.01}{#1}}
\newcommand{\PreprocessorTok}[1]{\textcolor[rgb]{0.56,0.35,0.01}{\textit{#1}}}
\newcommand{\RegionMarkerTok}[1]{#1}
\newcommand{\SpecialCharTok}[1]{\textcolor[rgb]{0.00,0.00,0.00}{#1}}
\newcommand{\SpecialStringTok}[1]{\textcolor[rgb]{0.31,0.60,0.02}{#1}}
\newcommand{\StringTok}[1]{\textcolor[rgb]{0.31,0.60,0.02}{#1}}
\newcommand{\VariableTok}[1]{\textcolor[rgb]{0.00,0.00,0.00}{#1}}
\newcommand{\VerbatimStringTok}[1]{\textcolor[rgb]{0.31,0.60,0.02}{#1}}
\newcommand{\WarningTok}[1]{\textcolor[rgb]{0.56,0.35,0.01}{\textbf{\textit{#1}}}}
\usepackage{graphicx}
\makeatletter
\def\maxwidth{\ifdim\Gin@nat@width>\linewidth\linewidth\else\Gin@nat@width\fi}
\def\maxheight{\ifdim\Gin@nat@height>\textheight\textheight\else\Gin@nat@height\fi}
\makeatother
% Scale images if necessary, so that they will not overflow the page
% margins by default, and it is still possible to overwrite the defaults
% using explicit options in \includegraphics[width, height, ...]{}
\setkeys{Gin}{width=\maxwidth,height=\maxheight,keepaspectratio}
% Set default figure placement to htbp
\makeatletter
\def\fps@figure{htbp}
\makeatother
\setlength{\emergencystretch}{3em} % prevent overfull lines
\providecommand{\tightlist}{%
  \setlength{\itemsep}{0pt}\setlength{\parskip}{0pt}}
\setcounter{secnumdepth}{-\maxdimen} % remove section numbering
\ifluatex
  \usepackage{selnolig}  % disable illegal ligatures
\fi

\title{Foundations of AI - Assignment 1}
\author{Garrett Ramela}
\date{9/23/2021}

\begin{document}
\maketitle

\hypertarget{r-markdown}{%
\subsection{R Markdown}\label{r-markdown}}

This is an R Markdown document. Markdown is a simple formatting syntax
for authoring HTML, PDF, and MS Word documents. For more details on
using R Markdown see \url{http://rmarkdown.rstudio.com}.

When you click the \textbf{Knit} button a document will be generated
that includes both content as well as the output of any embedded R code
chunks within the document. You can embed an R code chunk like this:

\begin{Shaded}
\begin{Highlighting}[]
\CommentTok{\# Ensure that your current working directory is set up properly.}
\CommentTok{\# setwd("\textasciitilde{}/Documents/George Washington University/Foundations of AI/Assignment 1")}

\CommentTok{\# Read the College.csv data set into RStudio.}
\NormalTok{college }\OtherTok{\textless{}{-}} \FunctionTok{read.csv}\NormalTok{(}\StringTok{\textquotesingle{}College.csv\textquotesingle{}}\NormalTok{) }

\CommentTok{\# Add a new column named rownames that R will not perform calculations on.}
\FunctionTok{rownames}\NormalTok{(college) }\OtherTok{=}\NormalTok{ college[ ,}\DecValTok{1}\NormalTok{]}
\FunctionTok{fix}\NormalTok{(college)}

\CommentTok{\# Remove the college column and only show the rownames column.}
\NormalTok{college }\OtherTok{=}\NormalTok{ college [ , }\SpecialCharTok{{-}}\DecValTok{1}\NormalTok{]}
\FunctionTok{fix}\NormalTok{(college)}

\CommentTok{\# Using the summary, head, and View functions to get a feel for the college data set.}
\FunctionTok{summary}\NormalTok{(college)}
\end{Highlighting}
\end{Shaded}

\begin{verbatim}
##    Private               Apps           Accept          Enroll    
##  Length:777         Min.   :   81   Min.   :   72   Min.   :  35  
##  Class :character   1st Qu.:  776   1st Qu.:  604   1st Qu.: 242  
##  Mode  :character   Median : 1558   Median : 1110   Median : 434  
##                     Mean   : 3002   Mean   : 2019   Mean   : 780  
##                     3rd Qu.: 3624   3rd Qu.: 2424   3rd Qu.: 902  
##                     Max.   :48094   Max.   :26330   Max.   :6392  
##    Top10perc       Top25perc      F.Undergrad     P.Undergrad     
##  Min.   : 1.00   Min.   :  9.0   Min.   :  139   Min.   :    1.0  
##  1st Qu.:15.00   1st Qu.: 41.0   1st Qu.:  992   1st Qu.:   95.0  
##  Median :23.00   Median : 54.0   Median : 1707   Median :  353.0  
##  Mean   :27.56   Mean   : 55.8   Mean   : 3700   Mean   :  855.3  
##  3rd Qu.:35.00   3rd Qu.: 69.0   3rd Qu.: 4005   3rd Qu.:  967.0  
##  Max.   :96.00   Max.   :100.0   Max.   :31643   Max.   :21836.0  
##     Outstate       Room.Board       Books           Personal   
##  Min.   : 2340   Min.   :1780   Min.   :  96.0   Min.   : 250  
##  1st Qu.: 7320   1st Qu.:3597   1st Qu.: 470.0   1st Qu.: 850  
##  Median : 9990   Median :4200   Median : 500.0   Median :1200  
##  Mean   :10441   Mean   :4358   Mean   : 549.4   Mean   :1341  
##  3rd Qu.:12925   3rd Qu.:5050   3rd Qu.: 600.0   3rd Qu.:1700  
##  Max.   :21700   Max.   :8124   Max.   :2340.0   Max.   :6800  
##       PhD            Terminal       S.F.Ratio      perc.alumni   
##  Min.   :  8.00   Min.   : 24.0   Min.   : 2.50   Min.   : 0.00  
##  1st Qu.: 62.00   1st Qu.: 71.0   1st Qu.:11.50   1st Qu.:13.00  
##  Median : 75.00   Median : 82.0   Median :13.60   Median :21.00  
##  Mean   : 72.66   Mean   : 79.7   Mean   :14.09   Mean   :22.74  
##  3rd Qu.: 85.00   3rd Qu.: 92.0   3rd Qu.:16.50   3rd Qu.:31.00  
##  Max.   :103.00   Max.   :100.0   Max.   :39.80   Max.   :64.00  
##      Expend        Grad.Rate     
##  Min.   : 3186   Min.   : 10.00  
##  1st Qu.: 6751   1st Qu.: 53.00  
##  Median : 8377   Median : 65.00  
##  Mean   : 9660   Mean   : 65.46  
##  3rd Qu.:10830   3rd Qu.: 78.00  
##  Max.   :56233   Max.   :118.00
\end{verbatim}

\begin{Shaded}
\begin{Highlighting}[]
\FunctionTok{head}\NormalTok{(college)}
\end{Highlighting}
\end{Shaded}

\begin{verbatim}
##                              Private Apps Accept Enroll Top10perc Top25perc
## Abilene Christian University     Yes 1660   1232    721        23        52
## Adelphi University               Yes 2186   1924    512        16        29
## Adrian College                   Yes 1428   1097    336        22        50
## Agnes Scott College              Yes  417    349    137        60        89
## Alaska Pacific University        Yes  193    146     55        16        44
## Albertson College                Yes  587    479    158        38        62
##                              F.Undergrad P.Undergrad Outstate Room.Board Books
## Abilene Christian University        2885         537     7440       3300   450
## Adelphi University                  2683        1227    12280       6450   750
## Adrian College                      1036          99    11250       3750   400
## Agnes Scott College                  510          63    12960       5450   450
## Alaska Pacific University            249         869     7560       4120   800
## Albertson College                    678          41    13500       3335   500
##                              Personal PhD Terminal S.F.Ratio perc.alumni Expend
## Abilene Christian University     2200  70       78      18.1          12   7041
## Adelphi University               1500  29       30      12.2          16  10527
## Adrian College                   1165  53       66      12.9          30   8735
## Agnes Scott College               875  92       97       7.7          37  19016
## Alaska Pacific University        1500  76       72      11.9           2  10922
## Albertson College                 675  67       73       9.4          11   9727
##                              Grad.Rate
## Abilene Christian University        60
## Adelphi University                  56
## Adrian College                      54
## Agnes Scott College                 59
## Alaska Pacific University           15
## Albertson College                   55
\end{verbatim}

\begin{Shaded}
\begin{Highlighting}[]
\FunctionTok{View}\NormalTok{(college)}

\CommentTok{\# Create a series of plots across the numerical variables within the college data set after}
\CommentTok{\# making the first column a numerical field.}
\NormalTok{college[, }\DecValTok{1}\NormalTok{] }\OtherTok{=} \FunctionTok{as.numeric}\NormalTok{(}\FunctionTok{factor}\NormalTok{(college[, }\DecValTok{1}\NormalTok{]))}
\FunctionTok{pairs}\NormalTok{(college[,}\DecValTok{1}\SpecialCharTok{:}\DecValTok{10}\NormalTok{])}
\end{Highlighting}
\end{Shaded}

\includegraphics{Assignment-1---Garrett-Ramela---RMarkdown_files/figure-latex/unnamed-chunk-1-1.pdf}

\begin{Shaded}
\begin{Highlighting}[]
\CommentTok{\# Use the boxplot() function to produce side{-}by{-}side box plots of Outstate versus Private}
\FunctionTok{boxplot}\NormalTok{(college}\SpecialCharTok{$}\NormalTok{Outstate }\SpecialCharTok{\textasciitilde{}}\NormalTok{ college}\SpecialCharTok{$}\NormalTok{Private,}
        \AttributeTok{xlab =} \StringTok{"No = Public School / Yes = Private School"}\NormalTok{,}
        \AttributeTok{ylab =} \StringTok{"Out{-}of{-}State Tuition Rate"}\NormalTok{,}
        \AttributeTok{col =} \StringTok{"Blue"}\NormalTok{)}
\end{Highlighting}
\end{Shaded}

\includegraphics{Assignment-1---Garrett-Ramela---RMarkdown_files/figure-latex/unnamed-chunk-1-2.pdf}

\begin{Shaded}
\begin{Highlighting}[]
\CommentTok{\# Creating variable Elite that includes the top 10 percent of students from high schools and}
\CommentTok{\# append the field to the existing college data set.}
\NormalTok{Elite }\OtherTok{=} \FunctionTok{rep}\NormalTok{(}\StringTok{"No"}\NormalTok{, }\FunctionTok{nrow}\NormalTok{(college))}
\NormalTok{Elite[college}\SpecialCharTok{$}\NormalTok{Top10perc }\SpecialCharTok{\textgreater{}} \DecValTok{50}\NormalTok{] }\OtherTok{=} \StringTok{"Yes"}
\NormalTok{Elite }\OtherTok{=} \FunctionTok{as.factor}\NormalTok{(Elite)}
\NormalTok{college }\OtherTok{=} \FunctionTok{data.frame}\NormalTok{(college, Elite)}

\CommentTok{\# Print a summary of the new college data set with the Elite column.}
\FunctionTok{summary}\NormalTok{(college)}
\end{Highlighting}
\end{Shaded}

\begin{verbatim}
##     Private           Apps           Accept          Enroll       Top10perc    
##  Min.   :1.000   Min.   :   81   Min.   :   72   Min.   :  35   Min.   : 1.00  
##  1st Qu.:1.000   1st Qu.:  776   1st Qu.:  604   1st Qu.: 242   1st Qu.:15.00  
##  Median :2.000   Median : 1558   Median : 1110   Median : 434   Median :23.00  
##  Mean   :1.727   Mean   : 3002   Mean   : 2019   Mean   : 780   Mean   :27.56  
##  3rd Qu.:2.000   3rd Qu.: 3624   3rd Qu.: 2424   3rd Qu.: 902   3rd Qu.:35.00  
##  Max.   :2.000   Max.   :48094   Max.   :26330   Max.   :6392   Max.   :96.00  
##    Top25perc      F.Undergrad     P.Undergrad         Outstate    
##  Min.   :  9.0   Min.   :  139   Min.   :    1.0   Min.   : 2340  
##  1st Qu.: 41.0   1st Qu.:  992   1st Qu.:   95.0   1st Qu.: 7320  
##  Median : 54.0   Median : 1707   Median :  353.0   Median : 9990  
##  Mean   : 55.8   Mean   : 3700   Mean   :  855.3   Mean   :10441  
##  3rd Qu.: 69.0   3rd Qu.: 4005   3rd Qu.:  967.0   3rd Qu.:12925  
##  Max.   :100.0   Max.   :31643   Max.   :21836.0   Max.   :21700  
##    Room.Board       Books           Personal         PhD        
##  Min.   :1780   Min.   :  96.0   Min.   : 250   Min.   :  8.00  
##  1st Qu.:3597   1st Qu.: 470.0   1st Qu.: 850   1st Qu.: 62.00  
##  Median :4200   Median : 500.0   Median :1200   Median : 75.00  
##  Mean   :4358   Mean   : 549.4   Mean   :1341   Mean   : 72.66  
##  3rd Qu.:5050   3rd Qu.: 600.0   3rd Qu.:1700   3rd Qu.: 85.00  
##  Max.   :8124   Max.   :2340.0   Max.   :6800   Max.   :103.00  
##     Terminal       S.F.Ratio      perc.alumni        Expend     
##  Min.   : 24.0   Min.   : 2.50   Min.   : 0.00   Min.   : 3186  
##  1st Qu.: 71.0   1st Qu.:11.50   1st Qu.:13.00   1st Qu.: 6751  
##  Median : 82.0   Median :13.60   Median :21.00   Median : 8377  
##  Mean   : 79.7   Mean   :14.09   Mean   :22.74   Mean   : 9660  
##  3rd Qu.: 92.0   3rd Qu.:16.50   3rd Qu.:31.00   3rd Qu.:10830  
##  Max.   :100.0   Max.   :39.80   Max.   :64.00   Max.   :56233  
##    Grad.Rate      Elite    
##  Min.   : 10.00   No :699  
##  1st Qu.: 53.00   Yes: 78  
##  Median : 65.00            
##  Mean   : 65.46            
##  3rd Qu.: 78.00            
##  Max.   :118.00
\end{verbatim}

\begin{Shaded}
\begin{Highlighting}[]
\CommentTok{\# Use the boxplot() function to produce side{-}by{-}side box plots of Outstate versus the new Elite variable.}
\FunctionTok{boxplot}\NormalTok{(college}\SpecialCharTok{$}\NormalTok{Outstate }\SpecialCharTok{\textasciitilde{}}\NormalTok{ college}\SpecialCharTok{$}\NormalTok{Elite,}
        \AttributeTok{xlab =} \StringTok{"No = Not Elite School / Yes = Elite School"}\NormalTok{,}
        \AttributeTok{ylab =} \StringTok{"Out{-}of{-}State Tuition Rate"}\NormalTok{,}
        \AttributeTok{col =} \StringTok{"Blue"}\NormalTok{)}
\end{Highlighting}
\end{Shaded}

\includegraphics{Assignment-1---Garrett-Ramela---RMarkdown_files/figure-latex/unnamed-chunk-1-3.pdf}

\begin{Shaded}
\begin{Highlighting}[]
\CommentTok{\# Print a series of six (6) histrograms showing distributions of applicants, accepted students,}
\CommentTok{\# percentage of PhDs, student/faculty ratio, percent of alumni who donate, and graduation rate.}
\FunctionTok{par}\NormalTok{(}\AttributeTok{mfrow =} \FunctionTok{c}\NormalTok{(}\DecValTok{2}\NormalTok{, }\DecValTok{3}\NormalTok{))}
\FunctionTok{hist}\NormalTok{(college}\SpecialCharTok{$}\NormalTok{Apps,}
     \AttributeTok{main =} \StringTok{"Student Applicants"}\NormalTok{,}
     \AttributeTok{xlab =} \StringTok{"Student Applicants"}\NormalTok{,}
     \AttributeTok{col =} \StringTok{"Blue"}\NormalTok{,}
     \AttributeTok{breaks =} \DecValTok{50}\NormalTok{)}
\FunctionTok{hist}\NormalTok{(college}\SpecialCharTok{$}\NormalTok{Accept,}
     \AttributeTok{main =} \StringTok{"Accepted Students"}\NormalTok{,}
     \AttributeTok{xlab =} \StringTok{"Accepted Students"}\NormalTok{,}
     \AttributeTok{col =} \StringTok{"Blue"}\NormalTok{,}
     \AttributeTok{breaks =} \DecValTok{50}\NormalTok{)}
\FunctionTok{hist}\NormalTok{(college}\SpecialCharTok{$}\NormalTok{PhD,}
     \AttributeTok{main =} \StringTok{"PhD Faculty Percentage"}\NormalTok{,}
     \AttributeTok{xlab =} \StringTok{"PhD Faculty Percentage"}\NormalTok{,}
     \AttributeTok{col =} \StringTok{"Blue"}\NormalTok{,}
     \AttributeTok{breaks =} \DecValTok{50}\NormalTok{)}
\FunctionTok{hist}\NormalTok{(college}\SpecialCharTok{$}\NormalTok{S.F.Ratio,}
     \AttributeTok{main =} \StringTok{"Student/Faculty Ratio"}\NormalTok{,}
     \AttributeTok{xlab =} \StringTok{"Student/Faculty Ratio"}\NormalTok{,}
     \AttributeTok{col =} \StringTok{"Blue"}\NormalTok{,}
     \AttributeTok{breaks =} \DecValTok{50}\NormalTok{)}
\FunctionTok{hist}\NormalTok{(college}\SpecialCharTok{$}\NormalTok{perc.alumni,}
     \AttributeTok{main =} \StringTok{"Donating Alumni Percentage"}\NormalTok{,}
     \AttributeTok{xlab =} \StringTok{"Donating Alumni Percentage"}\NormalTok{,}
     \AttributeTok{col =} \StringTok{"Blue"}\NormalTok{,}
     \AttributeTok{breaks =} \DecValTok{50}\NormalTok{)}
\FunctionTok{hist}\NormalTok{(college}\SpecialCharTok{$}\NormalTok{Grad.Rate,}
     \AttributeTok{main =} \StringTok{"Graduation Rate"}\NormalTok{,}
     \AttributeTok{xlab =} \StringTok{"Graduation Rate"}\NormalTok{,}
     \AttributeTok{col =} \StringTok{"Blue"}\NormalTok{,}
     \AttributeTok{breaks =} \DecValTok{50}\NormalTok{)}
\end{Highlighting}
\end{Shaded}

\includegraphics{Assignment-1---Garrett-Ramela---RMarkdown_files/figure-latex/unnamed-chunk-1-4.pdf}

\begin{Shaded}
\begin{Highlighting}[]
\CommentTok{\# Use the plot() function to produce side{-}by{-}side box plots of Outstate versus Private}
\FunctionTok{library}\NormalTok{(ggplot2)}
\FunctionTok{ggplot}\NormalTok{(college, }\FunctionTok{aes}\NormalTok{(}\AttributeTok{x =}\NormalTok{ S.F.Ratio, }\AttributeTok{y =}\NormalTok{ Outstate)) }\SpecialCharTok{+}
  \FunctionTok{geom\_point}\NormalTok{(}\FunctionTok{aes}\NormalTok{(}\AttributeTok{color =}\NormalTok{ Private, }\AttributeTok{shape =}\NormalTok{ Elite)) }\SpecialCharTok{+}
  \FunctionTok{geom\_smooth}\NormalTok{(}\AttributeTok{method =}\NormalTok{ lm) }\SpecialCharTok{+}
  \FunctionTok{labs}\NormalTok{(}\AttributeTok{title =} \StringTok{"Student/Faculty Ratio Effect on Tuition"}\NormalTok{,}
       \AttributeTok{subtitle =} \StringTok{"Grouped Across Public \& Public Schools"}\NormalTok{,}
       \AttributeTok{x =} \StringTok{"Student/Faculty Ratio"}\NormalTok{,}
       \AttributeTok{y =} \StringTok{"Out{-}of{-}State Tuition Rate"}\NormalTok{,}
       \AttributeTok{color =} \StringTok{"Private School"}\NormalTok{,}
       \AttributeTok{shape =} \StringTok{"Elite School"}\NormalTok{)}
\end{Highlighting}
\end{Shaded}

\begin{verbatim}
## `geom_smooth()` using formula 'y ~ x'
\end{verbatim}

\includegraphics{Assignment-1---Garrett-Ramela---RMarkdown_files/figure-latex/unnamed-chunk-1-5.pdf}

\begin{Shaded}
\begin{Highlighting}[]
\FunctionTok{ggplot}\NormalTok{(college, }\FunctionTok{aes}\NormalTok{(}\AttributeTok{x =}\NormalTok{ PhD, }\AttributeTok{y =}\NormalTok{ Outstate)) }\SpecialCharTok{+}
  \FunctionTok{geom\_point}\NormalTok{(}\FunctionTok{aes}\NormalTok{(}\AttributeTok{color =}\NormalTok{ Private, }\AttributeTok{shape =}\NormalTok{ Elite)) }\SpecialCharTok{+}
  \FunctionTok{geom\_smooth}\NormalTok{(}\AttributeTok{method =}\NormalTok{ lm) }\SpecialCharTok{+}
  \FunctionTok{labs}\NormalTok{(}\AttributeTok{title =} \StringTok{"Faculty PhD Percentage Effect on Tuition"}\NormalTok{,}
       \AttributeTok{subtitle =} \StringTok{"Grouped Across Public \& Public Schools"}\NormalTok{,}
       \AttributeTok{x =} \StringTok{"Faculty with PhDs Percentage"}\NormalTok{,}
       \AttributeTok{y =} \StringTok{"Out{-}of{-}State Tuition Rate"}\NormalTok{,}
       \AttributeTok{color =} \StringTok{"Private School"}\NormalTok{,}
       \AttributeTok{shape =} \StringTok{"Elite School"}\NormalTok{)}
\end{Highlighting}
\end{Shaded}

\begin{verbatim}
## `geom_smooth()` using formula 'y ~ x'
\end{verbatim}

\includegraphics{Assignment-1---Garrett-Ramela---RMarkdown_files/figure-latex/unnamed-chunk-1-6.pdf}

\begin{Shaded}
\begin{Highlighting}[]
\FunctionTok{ggplot}\NormalTok{(college, }\FunctionTok{aes}\NormalTok{(}\AttributeTok{x =}\NormalTok{ perc.alumni, }\AttributeTok{y =}\NormalTok{ Outstate)) }\SpecialCharTok{+}
  \FunctionTok{geom\_point}\NormalTok{(}\FunctionTok{aes}\NormalTok{(}\AttributeTok{color =}\NormalTok{ Private, }\AttributeTok{shape =}\NormalTok{ Elite)) }\SpecialCharTok{+}
  \FunctionTok{geom\_smooth}\NormalTok{(}\AttributeTok{method =}\NormalTok{ lm) }\SpecialCharTok{+}
  \FunctionTok{labs}\NormalTok{(}\AttributeTok{title =} \StringTok{"Donating Alumni Percentage Effect on Tuition"}\NormalTok{,}
       \AttributeTok{subtitle =} \StringTok{"Grouped Across Public \& Public Schools"}\NormalTok{,}
       \AttributeTok{x =} \StringTok{"Donating Alumni Percentage"}\NormalTok{,}
       \AttributeTok{y =} \StringTok{"Out{-}of{-}State Tuition Rate"}\NormalTok{,}
       \AttributeTok{color =} \StringTok{"Private School"}\NormalTok{,}
       \AttributeTok{shape =} \StringTok{"Elite School"}\NormalTok{)}
\end{Highlighting}
\end{Shaded}

\begin{verbatim}
## `geom_smooth()` using formula 'y ~ x'
\end{verbatim}

\includegraphics{Assignment-1---Garrett-Ramela---RMarkdown_files/figure-latex/unnamed-chunk-1-7.pdf}

\end{document}
